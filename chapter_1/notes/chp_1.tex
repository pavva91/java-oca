\documentclass{article}

\author{Bob Smiths}
\title{Chapter 1 - Java Building Blocks}

\begin{document}

\maketitle

\section{Java Class Structure}
Java basic structure is composed of classes that are instatiated as objects.

\textbf{CLASSES} := Classes are the basic building blocks.

\textbf{OBJECT} := An object is a \underline{runtime instace} of a class in memory.

\subsection{Field and Methods}
Java classes have two primary elements: \textit{methods} and \textit{fields}

\textbf{PUBLIC} := Is used to signify that this method may be called from other classes.

\textbf{VOID} := void means that no value at all is returned.

\textbf{PARAMETER} := The information required as input in the method.

\textbf{METHOD SIGNATURE} := The full declarationn of a method.

\subsection{Classes vs. Files}
You can even put two classes in the same file. At most one of the classes in the file is allowed to be public.
The public class needs to match the filename.

\subsection{Writing a \textit{main()} Method}
\textbf{COMPILE JAVA CODE} := javac on .java source file.

\textbf{RUN COMPILED JAVA} := java on .class bytecode file (no file extension needed)
:w

\textbf{ACCESS MODIFIER} := Declares the method's level of exposure.

\textbf{\textit{static} KEYWORD} := Binds a method to tis class so it can be called by just the class name.

\textbf{ARRAY} := An array is a fixed-size list of items that are all of the same type. An array is represented by ``[]'' brackets.

\textbf{VARARGS} := The characters ``...'' are called varargs (variable argument lists)

\subsection{Understanding Package Declarations and Imports}
\textbf{PACKAGES} := Packages are logical groupings for classes.

\textbf{IMPORT STATEMENT} := The import statement tells the compiler which package to look in to find a class.

\textbf{PACKAGE NAMING} := The rules for package names are the same as for variable names.

\textbf{COMPILERS FIGURES OUT WHAT IS ACTUALLY NEEDED} := One might think that including so many classes slows down the program, but it doesn't. The compiler figures out what's actually needed.

\textbf{NAMING CONFLICTS} := One of the reasons for using packages is so that class names don't have to be unique across all of java.

\textbf{DEFAULT PACKAGE} := This is a special unnamed package that you should use only for throwaway code. Is the default package because there's no package name.

\subsection{Constructors}
The name of the constructor matches the name of the class and there's no return type.

The purpose of the constructor is to initialize fields.

For most classes, you don't have to code a constructor - the compiler will supply a ``do nothing'' default constructor for you.

\subsection{Instance Initializer Blocks}
\textbf{CODE BLOCK} := The code between the braces is called a \textit{code block}

\textbf{INSTANCE INITIALIZER} := Code blocks that appear outside a method.

\subsection{Order of Initialization}
The order of initialization is:
\begin{itemize}
	\item Fields and instance initializer blocks are run in the order in which they appear in the file.
	\item The constructor runs after all fields and instance initializer blocks have run.
\end{itemize}
NOTE: Order matters for the fields and blocks of code.

\subsection{Distinguishing Between Object References and Primitives}
Java applications contains two types of data:
\begin{enumerate}
	\item Primitive types
	\item Reference tyeps
\end{enumerate}

\subsection{Primitive Types}
Java has eight built-in data types, called \textit{primitive types}. These eight data types represent the building blocks for Java objects, because all Java objects are just a complex collection of these primitive data types:
\begin{itemize}
	\item boolean
	\item byte
	\item short
	\item int
	\item long
	\item float
	\item double
	\item char
\end{itemize}

\textbf{LITERAL} := When a number is present in hte code, it is called literal

\subsubsection{Java Bases}
Another way to specify numbers is to change the ``base''. Java allows you to specify digits in several formats:
\begin{itemize}
	\item octal (digits 0-7), which uses the number 0 as a prefix - e.g.: 017
	\item hexadecimal (digits 0-9 and letters A-F), which uses the number 0 followed by x of X as a prefix - e.g.: 0xF5
	\item binary (digits 0-1), which uses the number 0 followed by b or B as a prefix - e.g.: 0b1011
\end{itemize}

\subsubsection{Underscore in Numbers}
You can have underscores in numbers to make them easier to read.

\textbf{RULE} := You can add underscores anywhere except:
\begin{itemize}
	\item At the beginning of a literal
	\item At the end of a literal
	\item Right before a decimal point
	\item Right after a decimal point
\end{itemize}

\subsection{Reference Types}

A reference type refers to an object (an instace of a class). Unlike primitive types that hold their values in the memory where the variable is allocated, references do not hold the value of the object they refer to. Instead, a reference ``points'' to an object by sotring the memory address where the object is located.

\textbf{POINTER} := A reference ``points'' to an object by storing the memory address where the object is located.

\textbf{ASSIGN REFERENCE} := There are 2 ways to assign a value to a reference:
\begin{itemize}
	\item A reference can be assigned to another object of the same type.
	\item A reference can be assigned to a new object using the new keyword.
\end{itemize}

\subsubsection{Key Differences}
\begin{enumerate}
	\item\textbf{NULL ASSIGNMENT} : Reference Types can be assigned null, which means they do not currently refer to an object. Primitive types will give you a compiler error if you attempt to assign them null.

	\item\textbf{CALL METHODS} : Reference types can be used to call methods when they do not point to null. Primitives do not have methods declared on them.
	\item All the primitive types have lowercase type names. All classes that come with Java begin with uppercase.
\end{enumerate}

\subsection{Declaring and Initializing Variables}
\textbf{VARIABLE} := A variable is a name for a piece of memory that stores data.





\end{document}
