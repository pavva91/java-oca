\documentclass{article}

\author{Bob Smiths}
\title{Chapter 1 - Java Building Blocks}

\begin{document}

\maketitle

\section{Java Class Structure}
Java basic structure is composed of classes that are instatiated as objects.

\textbf{CLASSES} := Classes are the basic building blocks.

\textbf{OBJECT} := An object is a \underline{runtime instace} of a class in memory.

\subsection{Field and Methods}
Java classes have two primary elements: \textit{methods} and \textit{fields}

\textbf{PUBLIC} := Is used to signify that this method may be called from other classes.

\textbf{VOID} := void means that no value at all is returned.

\textbf{PARAMETER} := The information required as input in the method.

\textbf{METHOD SIGNATURE} := The full declarationn of a method.

\subsection{Classes vs. Files}
You can even put two classes in the same file. At most one of the classes in the file is allowed to be public.
The public class needs to match the filename.

\subsection{Writing a \textit{main()} Method}
\textbf{COMPILE JAVA CODE} := javac on .java source file.

\textbf{RUN COMPILED JAVA} := java on .class bytecode file (no file extension needed)
:w

\textbf{ACCESS MODIFIER} := Declares the method's level of exposure.

\textbf{\textit{static} KEYWORD} := Binds a method to tis class so it can be called by just the class name.

\textbf{ARRAY} := An array is a fixed-size list of items that are all of the same type. An array is represented by ``[]'' brackets.

\textbf{VARARGS} := The characters ``...'' are called varargs (variable argument lists)

\subsection{Understanding Package Declarations and Imports}





\end{document}
